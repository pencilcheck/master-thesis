\chapter{Related Work}
\label{c:intro}

\section{Component Based Middleware}

Distributed Systems such as Wireless Sensor Netweorks (WSNs) composed of
multiple resource constrained devices equipped with low-power radios, low-cost
sensors collaborating to perform tasks for applications in multiple discipline
such as habitat monitoring. However, typically WSN applications are huge in
scale, subject to unreliable network, high risk of node failures, and expected
to run unattended for a long period of time, several work on lightweight component
models have been proposed\cite{Costa2007}\cite{Gay2003}\cite{Coulson2008}.

% About LooCI and REMORA
\subsection{Lightweight Component Models}

Work such as LooCI by Hughes et. al.\cite{Hughes2012} and Remora by Taherkordi
et. al.\cite{Taherkordi2010a} have demonstrated the feasibility of
reconfigurable infrastructure with event-driven programming model, event
management system, dynamic-binding to reduce development overhead, simplify
abstractions, and easier, dynamic reconfiguration for distributed applications.

\section{Fault Tolerance in Distributed Systems}

Many systems has been proposed to provide fault tolerance in distributed
systems, as the possibilities partial failures is a fundamental characterisics
of distributed systems.
However, distributed systems such as Wireless Sensor Networks typically
involves a large number of resource-constrained devices equipped with various
hardware components such as microprocessors, sensors, memory, wireless
communication. An application rely on the collaboration among the sensor nodes in
the network to perform tasks.

% No single point of failure

In order to prevent a single point of failure in applications that would bring
application to a halt in the event of failure, 
one of the primary goals for most of the work for fault tolerance is to
eliminiate single
point of failure in an application, so replications, redundancies are
recognized to be a good model for tackling the problems mentioned earlier in
distributed systems.
% replication and redundancy
Replication and redundancy are both techniques that allow the system to
duplicate multiple copies of a specific system component such that in the
event of failure of one of the components, one of the other duplicated
components can take over to perform the same services or functionality that the
previous one provides.

\subsection{Service-Oriented Architecture}

Neumann et. al.\cite{Neumann2010} proposed a new redundancy infrastructure to
bring Service-oriented Architecture (SOA) to Wireless Sensor Networks (WSN).

SOA is an established approach to ease development of complex distributed
applications by encapsulating system compoennts into services, this allows a more
flexible way to construct and develop interactions in application.

% what they lack for solving the problems I face
However, these approaches don't work well with applications where
reconfigurability, interoperability, and heterogeneity are requirements.
Existing approaches do not consider reconfigurability in the
application level that might invalidate existing infrastructure built upon one
instance of applications in another new instance when redeployed.
Most existing approaches also are not efficient in providing reducing the
development overhead to simplify fault tolerance in existing applications.
