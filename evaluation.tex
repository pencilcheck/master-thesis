\cleardoublepage
\singlespacing
\chapter{EVALUATION \& RESULTS}
\label{c:evaluation}
\doublespacing\nointerlineskip

%This chapter presents evaluation. The models and algorithms are
%tested extensively on benchmarks, which is described in
%section~\ref{s:benchmarks}, and the results are discussed in
%section~\ref{s:results}

In order to evaluate the performance of our fault tolerance system, we have
introduced some metrics to test how well it perform, and whether after node
failures requirements could still be met under small network.

\begin{enumerate}
\item Whether the next node in strip take over after failure
\item Memory overhead for Strips
\item Message overhead for failure recovery, including reconfiguration
\item Time to recover from failure
\end{enumerate}

Whenever a node failed, the the next node in the strips it is carrying would
take over for the respective services each represents.
Time to recover measures the time it takes to recover from the time of failure
detection.

We measured system performance live by collecting data from sensor nodes while
running. Sensor nodes are programmed to send out their tracking data to
a central data sink at appropriate times such as after node initialization or
when the failure is resolved.

The application, fault tolerance policy, network topology are described in the
following sections.


\section{Application}

Application shown in Figure~\ref{fig:fbp-application} will be deployed.  There
will be four components: 

\begin{enumerate}
\item Numeric Controller is a user input device which outputs
a number from 0 to 255. Light Sensor is a photodetector sensor which detects the
level of light intensity. 
\item Threshold is a conditional function which takes two
inputs, Threshold and Value, and, depending on the Operator attribute, return
true if the Operator is set to GT (Greater Than) and the Value is higher than
the Threshold. 
\item Light Actuator is a relay intercepting the power source for
a light bulb, it has a property OnOff which turns on the light if it is set to
true, otherwise the light will be turned off.
\item Light Sensor is a sensor sensing the light intensity in the surrounding
  area.
\end{enumerate}


\section{Policy}

The component fault tolerance policy for the application is set with the
following parameters:

\begin{description}
  \item[Numeric Controller] \hfill \\
    Redundancy Level: 1\\
    Fault Detection Time: 2 sec\\
  \item[Light Sensor] \hfill \\
    Redundancy Level: 2\\
    Fault Detection Time: 2 sec\\
  \item[Threshold] \hfill \\
    Redundancy Level: 1\\
    Fault Detection Time: 2 sec\\
  \item[Light Actuator] \hfill \\
    Redundancy Level: 9\\
    Fault Detection Time: 2 sec\\
\end{description}

Since we set timeout at 2 times of heartbeat period, assuming the worst time to detect
failure takes the full length of fault detection time, the heartbeat
period is therefore 1 sec, which is one half of the fault detection time.

\section{Heartbeat Protocol Arrangement}

\begin{figure}[h!]
\caption{Heartbeat Protocol Arrangement}
\label{fig:heartbeat-protocol-arrangement}
\centering
    \includegraphics[width=\linewidth]{figures/heartbeat-protocol-arrangement}
\end{figure}

We deployed 10 nodes in a room in our test lab, which results in a fully
connected network. Therefore only one heartbeat chain loop is formed.

The heartbeat procotol arrangement is simple. Every node is sending heartbeat to
previous node except the first node, which sends to the last. For example, node
1 receives heartbeat message from node 2, node 2 receives heartbeat message from
node 3, etc. Figure~\ref{fig:heartbeat-protocol-arrangement} illustrates the
arrangement for this experiment.

\section{Hardware Platform}

\begin{figure}[h!]
\caption{An WuDevice}
\label{fig:wudevice}
\centering
    \includegraphics[width=\linewidth]{figures/wudevice}
\end{figure}

All boards are equipped with an Atmel ATmega1280-16AU 8-bit microcontroller with
4K of EEPROM and 64k of flash. The boards hardware design is based upon Arduino
hardware referenced design, in addition, every board has wires for mounting
multiple wireless protocol adapters such as ZWave, ZigBee. In the following
experiments, every board is only equipped with a ZWave adapter, and only
communicating through ZWave.  Every board is also pre-installed with a modified
version of NanoVM~\cite{Harbaum2006} called “NanoKong”~\cite{Su} that supports all the basic
WuKong framework protocols including the new additions from the work in the
previous chapter.  A PC with wireless access is dedicated for hosting the WuKong
Master software which is responsible for managing WuKong applications for the
whole system and serves as a mean to present an interface to the users.  Three
boards will be used in the experiments below. One of them is equipped with
a light sensor that returns a byte indicating the light level around the sensor.
The rest are equipped with a relay which each controls the power supply of
a lamp.  An additional board with the same hardware specification is used as
a gateway between the Master and the sensor network.

\section{Experimental Setup}

\begin{table}
\centering
\caption{Node setup}
\label{tbl:setup}
  \begin{tabular}{|l|l|}
  \hline
  \textbf{Node Id} & \textbf{Equipped resources} \\
  \hline
  1(2) & Light Actuator \\
  \hline
  2(4) & Numeric Controller, Threshold, Light Sensor \\
  \hline
  3(5) & Light Actuator \\
  \hline
  4(6) & Light Actuator \\
  \hline
  5(7) & Light Actuator, Light Sensor \\
  \hline
  6(10) & Light Actuator \\
  \hline
  7(12) & Light Actuator \\
  \hline
  8(13) & Light Actuator \\
  \hline
  9(14) & Light Actuator \\
  \hline
  10(15) & Light Actuator \\
  \hline
  \end{tabular}
\end{table}

Ten WuDevices are installed throughout our testbed. Every WuDevice will be
able to talk to each other directly forming a fully connected network. Eight of them are
equipped with light actuators. Two of them have light sensors. Only one of them
has user input device (Numeric Controller), and Threshold. We simulate a node
failure by removing power supply of a WuDevice. Every device communicates
wirelessly through ZWave adapter. The setup is shown in table~\ref{tbl:setup}

\section{Mapping results}

\begin{table}
\centering
\caption{Strips}
\label{tbl:mapping-result}
  \begin{tabular}{|l|l|}
  \hline
  \textbf{Application Component} & \textbf{Mapped nodes (strip)} \\
  \hline
  Numeric Controller & 2 \\
  \hline
  Light Sensor & 2, 5 \\
  \hline
  Light Actuator & 1, 3, 4, 5, 6, 7, 8, 9, 10 \\
  \hline
  Threshold & 2 \\
  \hline
  \end{tabular}
\end{table}

The result of the mapping and the strips are shown in
table~\ref{tbl:mapping-result}. Each row represents each component in the
application, where strips are ordered from the left.


%\section{Mapping method}

%Deployments with different strip ordering method, such as first-fit, last-fit,
%closest-fit, will be performed to compare their effects on system performance.

%As first-fit was introduced in eariler chapter at chapter~\ref{c:design}, the
%other methods that will be used in the experiment are introduced here.

%Last fit is exactly first fit but reversing the order at the end.
%Closest fit sorts the strips by the order of the histogram of number of unique
%capability a node has.


\section{Results}
\label{s:results}

\begin{figure}[h!]
\caption{Average recovery time and message overhead over 5 deployments for each
node failure as the first failure}
\centering
    \includegraphics[width=\linewidth]{figures/results-average-recovery-time-plus-message-overhead}
\label{fig:results-average-recovery-time-plus-message-overhead}
\end{figure}

\begin{table}
\centering
\caption{Strip memory overhead in bytes}
\label{tbl:results-memory-overhead-strip}
  \begin{tabular}{|l|l|}
  \hline
  \textbf{Application Component Strip} & \textbf{Memory size (bytes)} \\
  \hline
  Numeric Controller & 2 \\
  \hline
  Light Sensor & 2 \\
  \hline
  Light Actuator & 18 \\
  \hline
  Threshold & 2 \\
  \hline
  \end{tabular}
\end{table}

The results for memory overhead by strips before failures are consistent, as
each node address takes only one byte, the other byte is used in our unique
identification system to recognize wuobject on a node. All failovers in
deployments have been swift and correct.

The figure~\ref{fig:results-average-recovery-time-plus-message-overhead}
illustrates the average recovery time and message overhead over 5 deployments
for each node failure in Strip for Light Actuator as first failure in the
system. The first failure should on average takes the longest time and higher
message overhead compared to consequent failures. Therefore measuing the
performance for each node failure as first failure would give us how the system
would perform the worst overall. The results carried over even in different
strip orders for Light Actuator Strip since the ordering is just a matter of
permuting the results as shown in the results.

The recovery time for most nodes were averaging around 2500 milli-seconds; node
3 and node 6 are found out that their radios were a little defective (without
antenna) after the experiments therefore it took longer to complete the
recovery. It is clear that the results have shown were pretty consistent as
there were only a constant number of nodes that needed to contact to recovery
regardless of how many strips the node contained. The time it took is reasonable
given the small network.


% TODO:Just pasted, need revision
Deployed to a network with 10 nodes, a 5-component application, which is mostly
the maximum number of complexity a typical real-world application could be, can
operate if each node could dedicate at most 100 bytes of memory to strips
(assuming one byte node addressing), and equipped with radios and battery
capable of handling approximately 15 messages for reconfiguration messages per
failure. The requirements are reasonable since most embedded devices have at
least 4K of EEPROM to store strips, and have radios with throughputs of 40kbps.
As shown in the results, the recovery time is reasonable within 3 seconds.

The size of network is also feasible since ZWave wireless protocol supports
a network up to 232 nodes, which is pretty big for most real world deployments
in areas such as home automation. 

