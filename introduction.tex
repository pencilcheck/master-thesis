\chapter{Introduction}
\label{c:intro}

This chapter provides an overview of the thesis. First, we describe why
WSN deployment and maintenance are still difficult, and how 
an intelligent middleware which employs a new programming model which separates
design abstractions between high-level application design and low-level
hardware constructs could be addressing the challenge. Next, we describe
the needs for fault tolerance and why it is hard on such middleware. Lastly, we
describe our proposed solution to this problem.

\section{Motivation}

\subsection{Wireless Sensor Network Deployment is Hard}

Wireless sensor networks are areas filled with network of tiny, resource
limited sensors communicating wirelessly. Each sensor is capable of sensing the
enviornment in its proximity. Wireless sensor networks are employed in
a variety of applications ranging from home automation to millitary.

Sensor networks offer the ability to monitor real-world phenomena in detail and
at large scale by embedding devices into the environment. Deployment is
about setting up an sensor network in a real-world environment. Deployment is
a labor-intensize and cumbersome task since environmental influences or
loose program logic in code might trigger bugs or sensor failures that
degrade performance in any way that has not been observed during pre-deployment
testing in the lab.

The real world has strong influences of the function of a sensor network that
could change the quality of wireless communication links, and by putting
extreme physical strains on sensor nodes. Laboratory testbed or simulator can 
only model to a very limited extent of those influences.

There have been several reports on sensor network installations where they
encountered problems during their
deployment\cite{Barrenetxea2008}\cite{Polastre2004}\cite{Arora2004}\cite{Tateson2005}\cite{Padhy2005}\cite{Stoianov2007}\cite{Tolle2005}\cite{Werner-Allen2006a}.

Testbed in laboratory environment can still not model the full extents of the
influences a real world enviroment could do. Deployment still a big problem in
wireless sensor network applications.

\subsection{Maintaining WSN deployment}

% TODO: could also borrow(rephrase the introduction in related work for FT

% talk briefly about fault tolerance, what it is, why it is important, give some examples if possible

The possibilities of sensor failures is a fundamental characteristic of 
distributed applications. So it is critical for distributed applications to
have the ability to detect and recover from failures automatically, since not
only the sensor deployments are usually huge in scale, the deployment are to
run unattended for a long period of time.

There have been abundance of research in fault tolerance for wireless sensor
network deployments proposing low-level programming abstractions or framework
to implement redundancy or replication.\cite{Whitehouse}\cite{Neumann2010}

\subsection{Component based middleware for distributed applications}

% TODO: need to change the paragraphs below, the reaons is slightly different
% now

% reasons for software componentization
However, the increase in number, size and complexity in WSN applications makes
high-level programming an essential needs for development in WSN platforms.

This is supported by several reasons. Firstly, the diversity of hardware and
software for WSN platform is as diverse as the programming models for such
platforms\cite{Sugihara2008}. Secondly, existing programming models usually
sacrifice resource usage with efficiency, which is not suitable for tiny
sensors in sensor networks. Thirdly, existing programming models still forces
developers to learn low-level languages, which imposes an extra burden to
developers, and it goes to show when the reusability for those programming
models are low in existing applications.

% what is software componentization, what is component models
Software componentization, or lightweight component models, has been recognized to
tackle the concerns above. It brings several advantages over past
approaches with separate of concern, module reusability, de-coupling, late
binding.

The primary advantages of this approach is reconfigability, adaptability in 
applications like never before, since the high-level components and
low-level constructs are loosely decoupled and interpretted by the middleware, 
high-level application logic can be added functionalities and framework around
it without changing the application logic at all, and applications can adapt to
different hardware configurations without changing any internal logic.

In Intel-NTU Center Special Interests Group for Context Analysis and Management 
(SIGCAM), our team have been collaborating on a project, called WuKong, to develop 
an intelligent middleware for developing, and deploying machine-to-mahicne 
(M2M) applications. The main contribution of this project is to support
inlligent mapping from a high-level flow based program (FBP) to
self-identified, context-specific sensors in a target
environment\cite{Reijers}.






% misc not needed for now
\begin{comment}
heterogeneous sensor platform where the level of resource constrains would be
different as some are designed for doing very specific job (such as sensor
motes), some were designed for handling large traffic and relaying messages
(such as gateway), the sensor network will each be connected to different types
network interfaces, and will have different types of sensors connected. The
complexity involved grows exponentially high.

, subject to unreliable network, high risk of node failures, and expected
to run unattended for a long period of time, several work have been proposing
lightweight component models to ease the development and deployment for WSN
applications.

problem is to separate applications into components where each will be
encapsulated into a single unit of operation that could be mapped into
low-level hardware functions running on the sensor
nodes\cite{Hughes2012}\cite{Taherkordi2010}.

However, as most work in this field of research assume homogeneous sensor
platforms and application requirements, fault tolerant solution for one
application cannot be extended to other applications. The chance of reusability
in those solutions among applications is still low.

Advances in sensor, radio technology and programming tools in recent years
enables rapid development of complex WSN applications on heterogeneous sensor
platforms, however, as current architectures are built around the idea of
a long deployment cycle, rapid redeployment is nearly impossible execute.

By bringing component based middleware to the playground, we could decouple
application logic and actual implementation easily, fault tolerance can be
broken down into several components which can be reused in different stage of
the deployment.

\end{comment}

\section{Problem definition}

\subsection{Component Based Fault Tolerance System}

The development and deployment for a fault tolerant application is still
immature in most component based middleware. Even though components are modular
in providing reconfigability to applications, they are still not failure resistent
and cannot recover from failures. The problem is worsen when the number of
components increase in applications, the developers would still bear the burden
of manually programming the applications to ensure fault tolerance.

\section{Proposed Solution}

% an overview of what I propose, what could benefit from it
I propose to investigate how applications described in high-level flow based 
program language could translate to low level constructs to create a
fault tolerant, applications in WuKong. In detail, we investigate how
system components collaborate to achieve a common goal while satisfying
application requirements to achieve self-fault detection, self-fault diagnosis,
and self-fault recovery. 

% NEED REVIEW
With colleagues from the Intel-NTU Special Interests Group for Context Analysis 
and Management (SIGCAM) at National Taiwan University, I have developed a 
new intelligent fault tolerance system, called Fur (temp), as part of WuKong
project, an Intelligent Middleware for developing and deploying 
applications on distributed platforms. Our proposed system consists of agents
collaborating to simplify fault
tolerance development and to shorten deployment cycle for heterogeneous M2M
applications.

% NEED REVIEW
The frivolous nature of requirements in applications, and actual physical sensor 
environements, along with the hard to predict user priorities, each contributes
a unique challenge in its flavor to developing an adapatable fault tolerance 
solution.

Below are the list of areas that this work will address solutions in.

\begin{description}
\item[Intelligent Mapping] \hfill \\
% talk briefly about compilation, what it is, why it is important, give some examples if possible
Flow-Based Programming (FBP) Paradigm has been used in WuKong to enable loosely
coupling between high-level application logic and low-level hardware
constructs. WuKongs
also achieves late-binding to bridge the worlds together at the last stage of
deployment using a technique
known as intelligent mapping. Intelligent Mapping is a process in which
high-level application logics are broken down into components and then mapped
to appropriate nodes. Sensor Profile Framework (SPF) are used in mapping to
handle heterogeneous sensor platforms, thus applications can be successfully
converted into lower hardware constructs to generate low-level intermediate
code to deploy to the sensor network.

\item[Sensor Profile Framework] \hfill \\
Sensor Profile framework provides an high-level abstraction to
sensor capabilities to enable building more complex application
logic.\cite{Reijers}

\item[User Policy Framework] \hfill \\
Allowing user-friendly specification of application executive objectives, and
context-dependent management of system peformance.

\item[Group Communication Systems] \hfill \\
A group communication system deals protocols for
synchronizing group states among group members in an consistent manner.\cite{Birman2012c}.
When the applications are deployed to the sensor network, groups will be
formed to implement redundancy. For a group of low-power sensor to collaborate on
a common goal, along with high-level application requirements and Sensor
Profile Framework, a new group communication protocol is needed to support 
fault tolerence.

\end{description}

\section{Thesis Organization}

% an overview of subsequent chapters
Our work overlaps many diverse but interconnected domains, each topic being 
itself a subject of advanced research and abundant literature. The second chapter
gives a brief background overview of these domains. We start by describing 
wireless sensor networks. Then we go on to discuss fault tolerant design for
distributed systems, it's objectives and recent developments. Finally, we will
be talking about component model based middleware. The third chapter gives some
overview of related work. The following two chapters describe our work in fault
tolerance for WuKong system. We first give an overview of some essential
components in WuKong. Then we give a comprehensive description of our fault
tolerance system architecture. We conclude this chapter by highlighting how the
integration of those subsystems, through careful scrutiny, could bring to the
development of a fault tolerant WSN application. We then present two
experiments to evaluate the performance, correctness of each mechanism in the
following chapter.
%TODO:Case studies, experiments and evaluation
This thesis concludes with a summarization of our proposed system and future
work.
