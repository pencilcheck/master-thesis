\cleardoublepage
\singlespacing
\chapter{INTRODUCTION}
\label{c:intro}
\doublespacing\nointerlineskip

%This chapter provides an overview of the thesis. First, we describe why
%deployment and maintenance of M2M systems such as WSN are still difficult and then we introduce the reconfigurable fault tolerance system problem and approaches to solving this
%problem. Next, we describe some related work on a reconfigurable fault tolerant distributed system. Lastly, we present our proposed solution to this problem.

Failover mechanism is is essential in any distributed networks. It is especially
crucial for enabling Internet-of-Things (IoT) in the sense that applications
could "deploy-once, run forever." Where applications degrade gracefully under
failures.

%\section{Wireless Sensor Network Deployment is Hard}

Internet-of-Things is being realized in Machine-to-Machine (M2M) systems such as
wireless sensor networks (WSN).  WSNs are areas filled with network of tiny,
resource limited sensors communicating wirelessly. Each sensor is
capable of sensing the enviornment in its proximity. Wireless sensor
networks are typically employed in a variety of applications ranging
from home automation to millitary.

Sensor networks offer the ability to monitor real-world phenomena in detail and
at large scale by embedding devices into the environment. Deployment is
about setting up an sensor network in a real-world environment. Deployment is
a labor-intensize and cumbersome task since environmental influences or
loose program logic in code might trigger bugs or sensor failures that
degrade performance in any way that has not been observed during pre-deployment
testing in the lab.

The real world has strong influences of the function of a sensor network that
could change the quality of wireless communication links, and by putting
extreme physical strains on sensor nodes. Laboratory testbed or simulator can 
only model to a very limited extent of those influences.

There have been several reports on sensor network installations where they
encountered problems during their
deployment\cite{Barrenetxea2008,Polastre2004,Arora2004,Tateson2005,Padhy2005,Stoianov2007,Tolle2005,Werner-Allen2006a}.

Testbed in laboratory environment can still not model the full extents of the
influences a real world enviroment could do. Deployment still a big problem in
wireless sensor network applications.

However, Wireless Sensor Network (WSN) programming can only be performed by
experts, because of resource constraints and their high failure probatility.
Hence, transforming the paradigm of service-oriented architecture (SOA) to
sensor networks becomes an important research in the last few
years.~\cite{Koutsoukos2007,Hughes2012}

WuKong, a middleware for service-oriented M2M systems, has set out to allow
developers develop M2M applications with ease by introducing users to a new
programming paradigm called flow based programming (FBP) with automatic sensor
identifications, node configuration, and system re-configuration.~\cite{Reijers}

In this thesis, we present our fault tolerance system for WuKong.

\section{Goals} % was Problem Definition

Users of WuKong deploy applications written in the form of flow based
programming. Applications consist of a set of components linked together.  Each
component represents a service in network. When application is deployed, each
component is mapped to a service in the network. However, partial failures could
bring the whole system down, since each service depends on each other to
function. However, as users have drastically different requirements even for the
same applications, we don't force the system to do in our way, we allow user
preference to influence our system via user policy. 

The goal of this system is to provide failover for application components such
that it meets user's requirements through user policy. Thus the system also
needs to design a generally accessible user policy for fault tolerance such that
it is easy and intuitive to specify how much fault tolerance a component should
have. It is also important that the deployed system would be able to detect,
  recover from failures, and reconfigure the system autonomously. 


\section{Challenges}

The way WuKong presents and models resources and applications, along with user
policy and requirements, makes it a unique challenge: User requirements might
vary drastically from one to another. System has to embrace a variety of
top-level and low-level configurations so components would play along with each
other and eventually produce a working failover architecture for the network.
Encompassing a variety of devices with different form factors, memory storage,
radio bandwidth is also a distinct characteristics in IoT,
therefore the system and algorithmic design has to work among nodes
with different specifications. In addition, duplications can be
costly and redundant when devices are heterogeneous or versaile,
such that instead of replaced by a device counterpart, it would be
taken over by another local service peers.

%Distributed systems have some unique properties that makes this problem really hard is that communication between nodes are not reliable and the ordering of the messages received could be out of order or dropped. Therefore most existing work treat the problem as a consensus problem where, mostly a variant of the solution proposed by Leslie Lamport in his paper~\cite{Lamport2001} where acceptors need to come to a consensus for the value of a particular variable made by the proposers, and the learners will have to learn of that decision. Under the assumption of a finite state machine for every process, the ordering properties is really important among the acceptors so none of them could reach a different state thus deviating from consensus. However, we will show that this is unnecessary in our approach.

%It is a challenege designing a solution that is scalable and also pertaining to the limitations when typical distributed systems such as wireless sensor networks have tight resource constraints and usually deploy in large quantity which dooms the thought of storing or maintaining any additional states, resouces has to be used economically. 

\section{Approaches}

Previous work on the problems has not considered failover architecture for
multiple services where relationship among services are critical and which would
also be influenced by user policies pre-deployment. Furthermore, after every
failure recovery, none explained how existing services could find replaced
services in such environment. This thesis proposed a novel algorithm to achieve
recovery from failures by combining heartbeat protocol, for failure detection,
with Strips, which are used to maintain and track service redundancy. 

\section{Related Work}

% I don't if I really need related work if it is based on the technique I used
% Or work with the similar settings (WuKong??)

In WSNs, failover techniques for data replications, and service redundancy are
typically adapt for applications only with predefined services, fixed user
requirements and services with only end-to-end links: read/write to local data
storage. One example of such work with local storage for replications is
described in~\cite{Ratnasamy2002}. In this work, some distinguished storage
nodes are specified by Hash functions to collect data of certain types.
Redundancy is achieved by storing replicas directly on neighbors nodes. But
applications are still restricted to end-to-end links, since there is only one
type of links in the application.

In \cite{Piotrowski2009}, a dynamic replication approach for local data storage
where replicas are randomly distributed within a predefined replication range
influenced by the specific replica number and/or its density is presented. 
Another dynamic replication approach for local data storage where replicas are
selected based on the scoring system defined by several physical resource traits
from self-inspection that could reactivate services based on already collected
and generated data~\cite{Neumann2010}. However, this approach does not consider
applications with dynamic user requirements, services with more complex links.

\section{Thesis Organization}

% an overview of subsequent chapters
Our work overlaps many diverse but interconnected domains, each topic being
itself a subject of advanced research and abundant literature.
Chapter~\ref{c:intro} gives an introduction to the system goals and challenges
and outline the approach used to solve the problem. Chapter~\ref{c:background}
gives a brief background overview of the topic that this work based on.  We
start by describing IoT, M2M networks and WSNs. The chapter ends with an
overview of WuKong. Chapter~\ref{c:design} describes our system design for the
WuKong fault tolerance system. In this chapter we give detail description of our
method and algorithms, including user policy for fault tolerance, strips, and
reconfigurable redundancy architecture.  Chapter~\ref{c:evaluation} presents the
metrics we would use to measure system performance, which is followed by
evaluation of the results. Finally, chapter ~\ref{c:conclusion} presents some
conclusions of the work, list of contributions and future work.
