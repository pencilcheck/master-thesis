\cleardoublepage
\singlespacing
\chapter{CONCLUSION}
\label{c:conclusion}
\doublespacing\nointerlineskip

%Previous chapters have described the work that has been done on WuKong platform.
%It is useful to reflect on what has been accomplished and place them in the
%broader context of the more general fault tolerance problem as well as the
%specific contributions of this work.

\section{Discussion}

This thesis proposed a novel algorithm to achieve
recovery from failures by combining heartbeat protocol, for failure detection,
with Strips, which are used to maintain and track service redundancy.

We have presented a fault tolerance system able to provide failover for failed
services in service-oriented WSNs that comply with user policy requirements. We
have also described strip, a redundancy abstraction for service peers along with
distributed algorithms to synchronize strip views among members and
reconfigurate the network for the new structure to recover from node failure.
The system allows user intervention through means of user policy, which could
directly influence underlying system configurations and structure.

The developed methods add new and useful solutions to build a fault tolerant
system that could be reasoned easily and with a performance as expected on
average. This method makes an extension to other methods in terms of
completeness and complexity. It serves as a quick and easy solution
to provide practical fault tolerance for WuKong applications.

% TODO: summarize the results of the experiments here, etc
% e.g. The solutions developed for WuKong system performs very consistently, as
% the recovery time is always under two secs on average with a handful of
% sensors. However, occationally heartbeats could report erroneous failures and
% result in a quick false recovery, and left the partitioned nodes behind, but
% the system as a whole still function properly.

The experimental results have shown to be consistent and stable among first
failures in different rank of members in strip around 2.5 secs. The failover
have been successful in all deployments. It is also shown that the performance
degraded quickly when the hardware or the wireless communication quality
degraded. Therefore it is important that the network setup is as optimized as
possible. 

\section{Future Work}

% Probably move it to future work? Probably don't do it.
%Of course there are a lot of room for improvements. For example, if we want to
%go beyond this limitation of 232 nodes by Zwave, one of the ways we could do is
%to build another zwave network and have a routing agent to route messages
%between networks. And we would also want to handle network partitions. One of
%the possible research directions is to merge the partitions back to one if they
%come back together, or by using some arbitrary flags to indicate primary
%components in the network to eliminate redundant commanding network components.
%Nonetheless, it is also an opportunity to look into multi-hop networks since
%heartbeats protocols are designed with single-hop network in mind, whether we
%could change the protocol to handle multi-hop is also a big challenge a future
%research could take on.


We have shown a design for a reconfigurable fault tolerant system for WuKong.
Strips makes it really easy to describe a component system with redundancy for
heterogeneous services and devices. Nevertheless, there is still room for
improvements. This section will address some directions future research can take.

WuKong Fault Tolerance System did not consider for network partition. Network
partition occurs when a network of nodes got partitioned into two subnetworks where
none can detect each other for a period of time. One of the possible direction
is to create a more sophisticated failure model that could handle network partition.
In this thesis we assume failstop model where nodes, once dead, will not come
back. Therefore when network partition occurs, each part of the network would
not be able to recognize each other and would cause conflicts and confusions.

Our current heartbeat protocol is distributed and easy to construct, but it is
not shown to work under networks where messages are sent in multiple hops, since
the algorithm used to produce where each node should be sending heartbeat
messages does not consider the topology of the network. Heartbeat is sensitive
on latency, so if a heartbeat message was not received within tolerance period,
a failure event could occur and the node is suspected of failure and will
never come back. If the node is still alive, it would be treated as if it is
dead. And that will creates an artificial network isolation where a few nodes
are excluded from the network before of latency.

Current system can only allow the detector to handle one failure at a time. It
would be a desirable future research direction to investigate handling
consecutive node failures. One possible way is by storing ahead multiple nodes'
strips and heartbeat protocol data, such that when consecutive nodes failure
occurred the detector would be able to recover those services it backed up.
However there is a tradeoff on the memory a node could store and the number of
consecutive node failures a network could handle.

\begin{comment}
Niels suggested that I show that I am aware of such issue with determining
optimality for deployment which is not clear for WuKong yet, there are many
ways or metrics to optimize for, all I can do in this work is to identify some
tradeoffs certain deployment for fault tolerance could influence the system
with certain metrics.

Limits will be hard to define here
Niels:about the tradeoffs in determining the deployment from your fault
tolerance perspective
Penn:Remember what the prof told me, about policy, first fit, last fit, etc
\end{comment}

The optimization problem for application deployment is also an important element
in this system. This thesis didn't consider finding a optimal deployment for the
level of redundancy specified in the user policy. The problem of deploying
a specific distributed system onto a network structure typically consists of
mapping the components of the system onto the hosts of the network. The mapping
is subject to constraints. The constraints could be whether a node supports
certain service to host certain components, and how much communication overhead
would induce from the assignment to maintain consistency for the strips, and
from the perspective of WuKong, some components need to seaparate from other
components to achieve fault tolerance, and some needs to place together to
function properly.  Determining such an optimal deployment is a combinatorial
optimization problem, and combinatorial optimization problems generally
extremely challenging computationally. It is difficult to predict what will and
what will not work.  It is unlikely that a single approach will be effective on
all problems or instances of the same problems. As we also want the system to
come up with a solution within a time limit. So finding a good balance between
the quality of a solution of the time it takes to come up with a good enough
solution is critical.
