\cleardoublepage
\singlespacing
\chapter{CONCLUSION}
\label{c:conclusion}
\doublespacing\nointerlineskip

%Previous chapters have described the work that has been done on WuKong platform.
%It is useful to reflect on what has been accomplished and place them in the
%broader context of the more general fault tolerance problem as well as the
%specific contributions of this work.

\section{Discussion}

We have presented a fault tolerance primitive for WuKong able to configure itself
based on user policy and requirements. And it is able to recover the system from failed
services in service-oriented WSNs as shown in experimental results. We
have described strip, a redundancy abstraction for service peers. We also
described the distributed algorithms to recover from node failures with strip
views synchronization among members and service reconfiguration.

The developed methods add new and useful solutions to build a fault tolerant
system that could be reasoned easily and with a performance as expected. Our
system provides an extension to other solutions in terms of completeness and
complexity. It serves as a quick and easy solution to provide practical fault
tolerance for service-oriented applications.

% TODO: summarize the results of the experiments here
% e.g. The solutions developed for WuKong system performs very consistently, as
% the recovery time is always under two secs on average with a handful of
% sensors. However, occasionally heartbeats could report erroneous failures and
% result in a quick false recovery, and left the partitioned nodes behind, but
% the system as a whole still function properly.


\section{Future Work}

% Probably move it to future work? Probably don't do it.
%Of course there are a lot of room for improvements. For example, if we want to
%go beyond this limitation of 232 nodes by Zwave, one of the ways we could do is
%to build another Zwave network and have a routing agent to route messages
%between networks. And we would also want to handle network partitions. One of
%the possible research directions is to merge the partitions back to one if they
%come back together, or by using some arbitrary flags to indicate primary
%components in the network to eliminate redundant commanding network components.
%Nonetheless, it is also an opportunity to look into multi-hop networks since
%heartbeats protocols are designed with single-hop network in mind, whether we
%could change the protocol to handle multi-hop is also a big challenge a future
%research could take on.


We have shown a design for a reconfigurable fault tolerant primitive for WuKong.
Strips makes it really easy to describe a component system with redundancy for
heterogeneous services and devices. Nevertheless, there is still room for
improvements. This section will address some directions future research can take.

This study did not consider for network partition. Network
partition occurs when a network of nodes got partitioned into two subnetworks where
none can detect each other for a period of time. One of the possible direction
is to create a more sophisticated failure model that could handle network partition.
In this thesis we assume fail-stop model where nodes, once dead, will not come
back. Therefore when network partition occurs, each part of the network would
not be able to recognize each other and would cause conflicts and confusions.

Our current heartbeat protocol is distributed and easy to construct, but it is
not shown to work under networks where messages are sent in multiple hops, since
the algorithm used to produce where each node should be sending heartbeat
messages does not consider the topology of the network. Heartbeat is sensitive
on latency, so if a heartbeat message was not received within tolerance period,
a failure event could occur and the node is suspected of failure and will
never come back. If the node is still alive, it would be treated as if it is
dead. And that will creates an artificial network isolation where a few nodes
are excluded from the network before of latency.

Current system can only allow the detector to handle one failure at a time. It
would be a desirable future research direction to investigate handling
consecutive node failures. One possible way is by storing ahead multiple nodes'
strips and heartbeat protocol data, such that when consecutive nodes failure
occurred the detector would be able to recover those services it backed up.
However there is a tradeoff on the memory a node could store and the number of
consecutive node failures a network could handle.

The optimization problem for application deployment is also an important element
in this system. This thesis didn't consider finding a optimal deployment for the
level of redundancy specified in the user policy. The problem of deploying
a specific distributed system onto a network structure typically consists of
mapping the components of the system onto the hosts of the network. The mapping
is subject to constraints. The constraints could be whether a node supports
certain service to host certain components, and how much communication overhead
would induce from the assignment to maintain consistency for the strips, and
from the perspective of WuKong, some components need to separate from other
components to achieve fault tolerance, and some needs to place together to
function properly.  Determining such an optimal deployment is a combinatorial
optimization problem, and combinatorial optimization problems generally
extremely challenging computationally. It is unlikely that a single approach
will be effective on all problems or instances of the same problems. As we also
want the system to come up with a solution within a time limit. So finding
a good balance between the quality of a solution of the time it takes to come up
with a good enough solution is critical.
