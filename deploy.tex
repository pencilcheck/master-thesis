\cleardoublepage
\singlespacing
\chapter{Deployment of Reconfigurable Atomic Service for Component Objects}
\label{c:deploy}
\doublespacing\nointerlineskip

\begin{comment}
Niels:about the tradeoffs in determining the deployment from your fault
tolerance perspective
\end{comment}

This chapter discussed the tradeoffs in determining optimal deployment for fault tolerance system such as RASCO. First it gives an overview of why it is a problem that we need to reconfigure the underlying network in a larger scale and how it is a problem to the system in itself in the long run. Then we will describe the deployment problem in the next section. Lastly, we identify certain tradeoffs in deployments for fault tolerance that could influence the system in a big way.

\section{Reconfigurable Atomic Service for Component Object}

\begin{comment}
Niels suggested that I show that I am aware of such issue with determining
optimality for deployment which is not clear for WuKong yet, there are many
ways or metrics to optimize for, all I can do in this work is to identify some
tradeoffs certain deployment for fault tolerance could influence the system
with certain metrics.

Limits will be hard to define here
\end{comment}

RASCO allows for continuous and large changes in the underlying
network. Providing consistent shared objects in a dynamic network is what RASCO
is developed for. RASCO achieves this by introducing the use of strips. Strips
enable the system to track, maintain replicas and maintain consistency in the
presence of failures. However, the number of failures RASCO can handle is still
limited by the number of spare nodes that could provide required services. It is
possible after a large change in the network, none of the nodes providing
a service is there anymore, so it is impossible for the newly joined
nodes to tap in and take over. To handle this large and permanent changes,
WuKong supports a system progression framework that could handle such large
change in the underlying network with dynamic \emph{reconfiguration}. WuKong
reconfiguration reassigns associations between software components and hosts.
But it does not have a solution to what the new assignment would look like.
This deployment problem is a gap that should be considered.

In the case of RASCO where deployment has to be continuously adjusted, the
deployment problem is more of a online problem, so there is a tighter limit on
time available to compute the assignment and strips.

\section{Optimal deployment}

The problem of deploying a specific distributed system onto a network structure
typically consists of mapping the components of the system onto the hosts of
the network. The mapping is subject to constraints. In the case for RASCO, the
constraints are whether a node supports certain service to host certain
components, and how much communication overhead would induce from the
assignment to maintain consistency for the strips, and from the perspective of
WuKong, some components need to seaparate from other components to achieve
fault tolerance, and some needs to place together to function properly.

Determining such an optimal deployment is a \emph{combinatorial optimization}
problem. Combinatorial optimization problems are generally extremely
challenging computationally. However, the problem is ubiquitous as it happens
all around our lifes, whether it is companies trying to assign limited
resources to meet certain objectives, or institutions allocating resources to
its staff members to minimize cost, are examples of combinatorial optimization
problems. However it is difficult to predict what will and what will not work.
It is unlikely that a single approach will be effective on all problems or
instances of the same problems.

\section{Set Covering Problem}

% http://www.cs.dartmouth.edu/~ac/Teach/CS105-Winter05/Notes/wan-ba-notes.pdf

\subsection{Greedy Approximation Method}

\section{Hybrid Method}

\section{Discussion}
