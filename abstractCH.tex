\begin{abstractCH}

\setlength{\baselineskip}{1.5em}

% 爸的翻譯
對於具有 "部署一次,永遠運行" 概念的物聯網而言,容錯移轉是這類分散式服務導向網路的必備條件。當設備更換時或系統出狀況時,必須利用資源再規畫去達成容錯移轉的機制。系統在運作時,異質性的或多工性的設備之間若不僅是端對端的訊息傳輸時,不管是設備或是訊息的複製都是昂貴且累贅。特別是當設備某種服務故障時,可由另外一個有能力提供相同服務的同級設備接替其服務,而不一定要由相同設備取代。利用長帶來記錄一連串複製的服務訊息,每一個同級設備都保存一致的長帶記錄。結合常用於失敗偵測的心跳協定,系統由異常回復的機制可以藉由操控分析分散狀態的長帶來達成。使用Arduino mega 2560相容設備所做的實驗結果顯示,我們已經能夠使小型網路系統故障復原,較大的網路實驗則正在進行中。未來研究方向包括確認網路的可擴展性,網路磁碟分割 處理以及解決同步故障的問題。

\end{abstractCH}
