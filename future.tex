\chapter{Future Work}
\label{c:future}

% here I can start rambling about all the limitations and directions on how to
% improve them

\section{Group connection types}

\section{Mapping}

Q: Inclusion problem, how to solve it? How severe it is?

A: The problem is due to the low redundancy level of a compoennt which is
mapped to a node with other compoennts which have higher redundancy levels.
This problem does not always have a solution, it depends highly upon the
resources in the network, it is unavoidable. However, we are considering
possible solutions to address this problem in the future by making sure the
mapping results does not have this pattern and could generate possible
solutions to solve this problem by either increase the redundancy level of the
vulnerable component or purposefully avoid mapping to dangerous nodes that
would put the component to risk.

But the bigger question is, if compoent FT policy is enough to ensure
application durability, at the current state, if any component does not have
minimum redundancy level of 2, that application is in the risk of single point
of fialure, there would be needed a way to inspect whether the policy is fault
tolerance at the application level, or introducing several application level FT
policy and that is something that we could be looking into in our next
research work.

\section{Policy}

% talk briefly about policy, what it is, why it is important, give some examples if possible

When application are getting complex full with features and configurations, 
it is important to have a high level declarative configuration policy language 
to specify the control for features and control of their respective behaviors 
smoothly. I propose a high level policy for fault tolerance that could be
translated to low level application requirements.
