\begin{abstractEN}

% from Shao-Wen
Enabled failover for M2M platforms in a distributed manner
Dynamic resource management on distributed networks is a prerequisite to
enabling Internet-of-Things (IoT) in the sense of “deploy-once, run forever.”
Resource reconfiguration is inevitable to achieve failover mechanisms upon
replacement of devices or failure of services. It can be particularly
challenging when devices are heterogeneous or versatile, for which duplications
can be costly and redundant. Specifically, a device with a failed service shall
be taken over by another service peer, instead of a device counterpart. Strip is
introduced to store a list of peer services, and, within the network, each
service maintains a consistent view of strips. In combination with the heartbeat
protocol widely applies for failure detection, recovery from failure can be
achieved by manipulating strips in a distributed manner, without a centralized
repository. Experiments using Arduino Mega 2560 devices show that our approach
is capable of failover in small networks, whereas experiments in larger networks
are underway. Future extensions include addressing the scalability issue and
tackling simultaneous failures.

% my original
As the variety of sensors and actuators increase, applications for distributed systems are getting more complex. Repairment becomes difficult to perform manually. It is appealing to design a system that could achieve fault tolerance without require human intervention.

This thesis investigated this problem. There are many different techniques exist to solve this problem under different assumptions with varying guarantees, what works best for one instance of a problem does not always apply to the others. This thesis proposed a new technique to solve the problem under the assumption of stateless application and heterogeneous network with a varity of hardware specifications where some nodes could perform multiple services at a time. The technique we proposed does not require any strong message ordering properties, and it requires only linear message complexity

% Do I need to describe what this thesis will present??

\end{abstractEN}

\begin{comment}

\category{I2.10}{Computing Methodologies}{Artificial Intelligence --
Vision and Scene Understanding}

\terms{System, Policy}

\keywords{Component Architecture Middleware, Fault Tolerance}

\end{comment}
