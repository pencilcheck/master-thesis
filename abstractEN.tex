\begin{abstractEN}

% credit to Shao-Wen
%\textbf{Enabled failover for M2M platforms in a distributed manner}
\noindent Failover for service-oriented distributed networks is a prerequisite
to enabling Internet-of-Things (IoT) in the sense of “deploy-once, run forever.”
Resource reconfiguration is required to achieve failover mechanisms upon
replacement of devices or failure of services. It can be particularly
challenging when services in applications have more than end-to-end
transmissions between devices that are heterogeneous or versatile, for which
duplications can be costly and redundant. Specifically, a device with a failed
service shall be taken over by another service peer, instead of a device
counterpart to recover application as a whole. Strip is introduced to store
a list of duplicated services, and, each service peer maintains a consistent
view of strips. In combination with the heartbeat protocol which was widely
applied for failure detection, recovery from failure can be achieved by
manipulating strips in a distributed manner. Experiments using Arduino Mega 1280
compatible devices show that our approach is capable of failover in small
networks, whereas experiments in larger networks are underway. Future research
directions include addressing the scalability issue, network partitions and
tackling simultaneous failures.

\end{abstractEN}

\begin{comment}

%\category{I2.10}{Computing Methodologies}{Artificial Intelligence -- Vision and Scene Understanding}

\terms{System, Policy}

\keywords{Service-Oriented WSN, Fault Tolerance}

\end{comment}
